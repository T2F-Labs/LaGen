\documentclass[a4paper,11pt,xetex]{article}

% Load ifthen package for conditional compilation
\usepackage{ifthen}

% Cover selection mechanism
% Options: none (default), 1, 2, etc. - numbers correspond to cover files in covers directory
\newcommand{\covertype}{none}

% ========== BRAND IDENTITY SETTINGS ==========
% Customize these to match your organization's branding
\newcommand{\companyName}{Lagen}
\newcommand{\companyTagline}{Professional Document Systems}
\newcommand{\documentVersion}{1.0.3}
\newcommand{\documentStatus}{Final Release}
\newcommand{\documentClassification}{Internal Use}

% Import all configuration and styling from config.tex
% ========== FONT AND ENCODING ==========
% Instructions: These packages handle font setup and character encoding
% Keep these as-is unless you need different fonts
\usepackage{inputenc}
\usepackage[T1]{fontenc}
\usepackage{lmodern}
\usepackage{fontspec}
\usepackage{setspace}
\usepackage{xcolor}
\usepackage{morefloats}
\usepackage[table]{xcolor}
% Professional letter spacing for brand names
% \usepackage{letterspace} % Removed as fontspec will handle letterspacing

% Load microtype with XeTeX compatibility options
\usepackage[protrusion=true,expansion=false]{microtype}

% Dynamic font scaling for brand consistency
\usepackage{scalefnt}
% Replace "Inter" with your preferred font family if needed
\setmainfont{Inter}[
    UprightFont = *-Regular,
    BoldFont = *-SemiBold,
    ItalicFont = *-Italic,
    BoldItalicFont = *-SemiBoldItalic,
    LetterSpace = 3.0 % Base letter spacing for the font
]
\setsansfont{Inter}[
    UprightFont = *-Regular,
    BoldFont = *-SemiBold,
    ItalicFont = *-Italic,
    BoldItalicFont = *-SemiBoldItalic,
    LetterSpace = 3.0 % Base letter spacing for the font
]

% Define branded text macros with letterspacing using fontspec
\newcommand{\brandname}[1]{{\addfontfeatures{LetterSpace=12.0}\textsc{#1}}}
\newcommand{\productname}[1]{{\addfontfeatures{LetterSpace=8.0}\textbf{#1}}}

% ========== PAGE LAYOUT ==========
% Instructions: Adjust measurements to change margins and spacing
\usepackage{geometry}
\geometry{
    top=1.2in,
    bottom=1.2in,
    left=1.1in,
    right=1.1in,
    headheight=14pt,
    headsep=0.3in,
    footskip=0.4in
}

% Better float placement control
\usepackage{placeins}
% Ensure floats appear after their references
\usepackage{flafter}
% Bug fixes for LaTeX kernel
\usepackage{fixltx2e}

% ========== COLORS ==========
% Instructions: Modify these color definitions to match your brand/theme
\definecolor{primary}{HTML}{0F172A}        % Main text color - change as needed
\definecolor{secondary}{HTML}{334155}      % Secondary text color
\definecolor{accent}{HTML}{2563EB}         % Links and highlights
\definecolor{lightaccent}{HTML}{3B82F6}    % Lighter accent variant
\definecolor{success}{HTML}{059669}        % Success/positive elements
\definecolor{warning}{HTML}{D97706}        % Warning/attention elements
\definecolor{background}{HTML}{F8FAFC}     % Light background for boxes
\definecolor{border}{HTML}{E2E8F0}        % Borders and lines
\definecolor{textgray}{HTML}{64748B}      % Muted text
\definecolor{lightgray}{HTML}{F1F5F9}     % Table alternating rows
\definecolor{codebg}{HTML}{F8F8F8}        % Code block background
\definecolor{codecomment}{HTML}{6A737D}   % Code comments
\definecolor{codestring}{HTML}{032F62}    % Code strings
\definecolor{codekey}{HTML}{D73A49}       % Code keywords

% ========== TYPOGRAPHY ==========
% Instructions: These settings control paragraph spacing and text optimization
\usepackage{microtype}  % Advanced typography refinements
\usepackage{parskip}    % Paragraph spacing control
\usepackage{soul}       % Underlining, strikeout, spacing, etc.
\usepackage{ragged2e}   % Better text alignment
\usepackage{textcase}   % Advanced case changing
\usepackage{relsize}    % Relative font size adjustments

% Basic paragraph settings
\setlength{\parindent}{0pt}
\setlength{\parskip}{8pt plus 2pt minus 1pt}

% Customizable line spacing commands
\newcommand{\setSpacingSingle}{\setstretch{1.0}}
\newcommand{\setSpacingOneHalf}{\setstretch{1.5}}
\newcommand{\setSpacingDouble}{\setstretch{2.0}}
\newcommand{\customspacing}[1]{\setstretch{#1}}

% Text decoration commands (with unique names)
\newcommand{\textunderline}[1]{\underline{#1}}    % Underlined text
\newcommand{\textstrikeout}[1]{\st{#1}}           % Strikethrough text
\newcommand{\texthi}[2][yellow]{\colorbox{#1}{#2}}  % Highlighted text

% Tracking (letter spacing) adjustments
\newcommand{\wide}[1]{\so{#1}}               % Wider letter spacing
\newcommand{\wider}[1]{\sowide{#1}}          % Even wider spacing
\newcommand{\narrow}[1]{\sonormal{#1}}       % Condensed letter spacing

% Font size relative adjustments (avoiding conflict with relsize definitions)
\newcommand{\textsm}[1]{\textsmaller{#1}}    % Smaller text (alternate name)
\newcommand{\textlg}[1]{\textlarger{#1}}     % Larger text (alternate name)

% Case transformation with preservation of special characters
\newcommand{\uppercased}[1]{\MakeTextUppercase{#1}}
\newcommand{\lowercased}[1]{\MakeTextLowercase{#1}}

% Special text styles
\newcommand{\textshadow}[2][gray]{{%
  \setlength{\fboxsep}{0pt}%
  \colorbox{white}{%
    \makebox[0pt][l]{\color{#1}#2}%
    \makebox[0pt][l]{\hspace{0.2pt}\vspace{0.2pt}#2}%
    \makebox[0pt][l]{\color{white}\hspace{0.4pt}\vspace{0.4pt}#2}%
    \makebox[0pt][l]{\color{white}\hspace{0.6pt}\vspace{0.6pt}#2}%
    #2%
  }%
}}

% Text alignment with better justification
\newcommand{\justifiedtext}{\justify}
\newcommand{\leftalignedtext}{\raggedright}
\newcommand{\rightalignedtext}{\raggedleft}
\newcommand{\centeredtext}{\centering}

% Typography fine-tuning
\setlength{\emergencystretch}{3em}  % Helps with justification
\hyphenpenalty=500                  % Reduce hyphenation
\clubpenalty=10000                  % Prevent orphans (single line at bottom of page)
\widowpenalty=10000                 % Prevent widows (single line at top of page)

% Microtype advanced settings
\microtypesetup{
    activate=true,           % Activate protrusion
    final=true,              % Enable microtype even if draft mode
    factor=1100              % Factor for enhanced protrusion
    % All expansion/tracking/spacing features disabled via package options
}

% XeTeX-compatible alternative for letter spacing (instead of microtype's tracking)
\newcommand{\letterspace}[2][100]{{\addfontfeatures{LetterSpace=#1}#2}}

% Add note about XeTeX-specific alternatives
% Note: With XeTeX, use fontspec features for advanced typography:
% \addfontfeatures{LetterSpace=X} for letter spacing (X = value in 1/1000 em)
% \addfontfeatures{Scale=X} for font scaling
% \addfontfeatures{Kerning=X} for kerning (where X is "On" or "Off")

% Control font features
\newcommand{\withfeatures}[2]{{\addfontfeatures{#1}#2}}

% ========== ADVANCED TEXT FEATURES ==========
% Instructions: For elegant drop caps at section beginnings
\usepackage{lettrine}
% Configuration for lettrine (drop caps)
\renewcommand{\LettrineFontHook}{\color{accent}\bfseries}
\setcounter{DefaultLines}{3}
\renewcommand{\DefaultLoversize}{0.1}
\renewcommand{\DefaultLraise}{0}

% Instructions: For proper quotation handling and formatting
\usepackage[autostyle=true,german=quotes,english=american]{csquotes}
\MakeOuterQuote{"}

% Instructions: For better paragraph flow control
\usepackage[all]{nowidow}
\usepackage{needspace}

% ========== SECTION STYLING ==========
% Instructions: Customize section headings appearance here
\usepackage{titlesec}
\titleformat{\section}
    {\color{primary}\fontsize{16}{20}\bfseries\sffamily}
    {\color{accent}\thesection}
    {1em}
    {}
    [\vspace{2pt}\color{border}\hrule height 0.8pt\vspace{6pt}]

\titleformat{\subsection}
    {\color{secondary}\fontsize{14}{18}\bfseries\sffamily}
    {\color{accent}\thesubsection}
    {0.8em}
    {}
    [\vspace{4pt}]

\titleformat{\subsubsection}
    {\color{secondary}\fontsize{12}{16}\bfseries\sffamily}
    {\color{accent}\thesubsubsection}
    {0.6em}
    {}

\titlespacing*{\section}{0pt}{20pt plus 4pt minus 2pt}{8pt plus 2pt minus 2pt}
\titlespacing*{\subsection}{0pt}{16pt plus 3pt minus 2pt}{6pt plus 2pt minus 1pt}
\titlespacing*{\subsubsection}{0pt}{12pt plus 2pt minus 1pt}{4pt plus 1pt minus 1pt}

% ========== HEADERS AND FOOTERS ==========
% Instructions: Replace "Your Document Title" and footer text with your content
\usepackage{fancyhdr}
\usepackage{lastpage} % For "Page X of Y" formatting
% \usepackage{scrlayer-scrpage} % Removed to avoid conflicts with fancyhdr
\pagestyle{fancy}
\fancyhf{}
\renewcommand{\headrulewidth}{0.5pt}
\renewcommand{\footrulewidth}{0pt}
\fancyhead[L]{\color{textgray}\small\sffamily Your Document Title Here}
\fancyhead[R]{\color{textgray}\small\sffamily\thepage}
\fancyfoot[C]{\color{textgray}\footnotesize\sffamily Page \thepage{} of \pageref{LastPage} | Document Footer Text Here}

% ========== TABLE OF CONTENTS ==========
% Instructions: These settings style the table of contents
\usepackage{tocloft}
\renewcommand{\cftsecleader}{\cftdotfill{\cftdotsep}}
\renewcommand{\cftsubsecleader}{\cftdotfill{\cftdotsep}}
\setlength{\cftbeforesecskip}{6pt}
\setlength{\cftbeforesubsecskip}{3pt}
\renewcommand{\cftsecfont}{\color{primary}\bfseries\sffamily}
\renewcommand{\cftsubsecfont}{\color{secondary}\sffamily}
\renewcommand{\cftsecpagefont}{\color{accent}\bfseries\sffamily}
\renewcommand{\cftsubsecpagefont}{\color{accent}\sffamily}

% ========== ADVANCED REFERENCE MANAGEMENT ==========
% Instructions: For sophisticated bibliography management
\usepackage[
    backend=biber,
    style=authoryear,
    sorting=nyt,
    maxbibnames=99,
    giveninits=true
]{biblatex}
% If you have a bibliography file, uncomment and specify it here:
% \addbibresource{references.bib}

% Instructions: For page-aware references and cross-references
% IMPORTANT: the order matters! hyperref first, then nameref, then varioref, and cleveref LAST
\usepackage{hyperref}
\hypersetup{
    colorlinks=true,
    linkcolor=accent,
    urlcolor=lightaccent,
    citecolor=accent,
    filecolor=accent,
    bookmarks=true,
    bookmarksopen=true,
    bookmarksnumbered=true,
    pdfstartview=FitH,
    pdfpagelayout=SinglePage,
    pdftitle={Your PDF Title Here},
    pdfauthor={Your Name Here},
    pdfsubject={Document Subject Here},
    pdfkeywords={keyword1, keyword2, keyword3}
}
\usepackage{nameref} % Reference section names, not just numbers
\usepackage{varioref} % For page-aware references
\labelformat{figure}{Figure~#1}
\labelformat{table}{Table~#1}
\labelformat{equation}{Equation~#1}
\labelformat{section}{Section~#1}

% ========== MATHEMATICAL TYPESETTING ==========
% Essential enhanced math environments
\usepackage{amsmath}
% Extended mathematical symbol collection
\usepackage{amssymb}
% Fix common mathematical typography issues
\usepackage{fixmath}
% Proper formatting of numbers and units
\usepackage{siunitx}

% Configure siunitx for consistent number formatting
\sisetup{
    group-separator={,},
    group-minimum-digits=4,
    detect-weight=true,
    detect-family=true
}

% Define common units and constants
\DeclareSIUnit\USD{\$}
\DeclareSIUnit\hour{h}
\DeclareSIUnit\year{yr}

% ========== ENHANCED MATHEMATICAL TYPESETTING ==========
% Advanced math tools and additional environments
\usepackage{mathtools}
% Enhanced equation highlighting and boxes
\usepackage{empheq}
% Advanced cases environments
\usepackage{cases}
% Theorem environments for mathematical statements
\usepackage{amsthm}
% Advanced symbol sets
\usepackage{stmaryrd}
\usepackage{textcomp}
% Physics notation package for differential operators, brackets, etc.
\usepackage{physics}
% Tensor notation
\usepackage{tensor}
% Quantum mechanics notation
\usepackage{braket}
% Cross out terms in equations
\usepackage{cancel}

% Theorem environments setup
\theoremstyle{definition}
\newtheorem{definition}{Definition}[section]
\newtheorem{theorem}{Theorem}[section]
\newtheorem{lemma}[theorem]{Lemma}
\newtheorem{corollary}[theorem]{Corollary}
\newtheorem{proposition}[theorem]{Proposition}
\newtheorem{example}{Example}[section]

\theoremstyle{remark}
\newtheorem{remark}{Remark}[section]
\newtheorem{note}{Note}[section]

% Custom colors for math highlighting
\definecolor{mathblue}{RGB}{0,82,155}
\definecolor{mathred}{RGB}{204,0,0}
\definecolor{mathgreen}{RGB}{0,128,0}
\definecolor{mathpurple}{RGB}{128,0,128}

% Common mathematical sets
\newcommand{\R}{\mathbb{R}}
\newcommand{\C}{\mathbb{C}}
\newcommand{\N}{\mathbb{N}}
\newcommand{\Z}{\mathbb{Z}}
\newcommand{\Q}{\mathbb{Q}}
\newcommand{\F}{\mathbb{F}}

% Enhanced derivatives and differential operators
\renewcommand{\d}{\mathrm{d}}
\newcommand{\dt}{\frac{\d}{\d t}}
\newcommand{\dx}{\frac{\d}{\d x}}
\newcommand{\dy}{\frac{\d}{\d y}}
\newcommand{\dz}{\frac{\d}{\d z}}
\newcommand{\prt}{\partial}
\newcommand{\pdx}[1]{\frac{\partial #1}{\partial x}}
\newcommand{\pdy}[1]{\frac{\partial #1}{\partial y}}
\newcommand{\pdz}[1]{\frac{\partial #1}{\partial z}}
\newcommand{\pdt}[1]{\frac{\partial #1}{\partial t}}

% Vector calculus operators
\providecommand{\grad}{\nabla}
\providecommand{\divg}{\nabla \cdot}
\providecommand{\curl}{\nabla \times}
\providecommand{\lapl}{\nabla^2}

% Enhanced formatting for vectors and matrices
\newcommand{\vect}[1]{\mathbf{#1}}
\newcommand{\mat}[1]{\mathbf{#1}}
\newcommand{\T}{\mathsf{T}} % Transpose

% Probability and statistics notation
\newcommand{\prob}[1]{\mathrm{P}\left(#1\right)}
\newcommand{\E}[1]{\mathrm{E}\left[#1\right]}
\renewcommand{\var}[1]{\mathrm{Var}\left(#1\right)}
\newcommand{\cov}[1]{\mathrm{Cov}\left(#1\right)}
\newcommand{\normal}{\mathcal{N}}
\newcommand{\uniform}{\mathcal{U}}

% Specialized math operators
\providecommand{\tr}{\operatorname{tr}}
\providecommand{\diag}{\operatorname{diag}}
\providecommand{\rank}{\operatorname{rank}}
\newcommand{\sign}{\operatorname{sign}}
\providecommand{\lcm}{\operatorname{lcm}}
\providecommand{\gcd}{\operatorname{gcd}}

% Box commands for important equations
\newcommand{\boxedeq}[1]{\begin{empheq}[box=\fbox]{equation}#1\end{empheq}}
\newcommand{\colorboxedeq}[2]{\begin{empheq}[box={\fcolorbox{#1}{white}}]{equation}#2\end{empheq}}

% ========== PAGE BREAK CONTROL ==========
% Instructions: For better page break control
\usepackage{afterpage}

% ========== BOXES AND CALLOUTS ==========
% Instructions: Required TikZ libraries for shadows in boxes
\usepackage{cleveref}
% Instructions: cleveref must be loaded AFTER hyperref, nameref, and varioref
\crefname{figure}{Figure}{Figures}
\crefname{table}{Table}{Tables}
\crefname{equation}{Equation}{Equations}
\crefname{section}{Section}{Sections}

\usepackage{tikz}
\usetikzlibrary{shadows}

% Instructions: Three types of callout boxes are defined below
\usepackage[most]{tcolorbox}

% Info box style - use for general information
\newtcolorbox{infobox}{
    colback=background,
    colframe=accent,
    coltext=primary,
    boxrule=1pt,
    arc=3pt,
    left=12pt,
    right=12pt,
    top=8pt,
    bottom=8pt,
    enhanced,
    drop shadow,
    fonttitle=\bfseries\sffamily\color{accent},
    title style={left color=accent!10, right color=accent!5},
    attach boxed title to top left={xshift=8pt, yshift=-2pt}
}

% Alert box style - use for cautions and important notes
\newtcolorbox{alertbox}{
    colback=warning!5,
    colframe=warning,
    coltext=primary,
    boxrule=1pt,
    arc=3pt,
    left=12pt,
    right=12pt,
    top=8pt,
    bottom=8pt,
    enhanced,
    drop shadow
}

% Success box style - use for positive highlights
\newtcolorbox{successbox}{
    colback=success!5,
    colframe=success,
    coltext=primary,
    boxrule=1pt,
    arc=3pt,
    left=12pt,
    right=12pt,
    top=8pt,
    bottom=8pt,
    enhanced,
    drop shadow
}

% ========== LOGO-ENHANCED BOXES ==========
\usepackage{bclogo}

% ========== LISTS ==========
% Instructions: These settings control list appearance and spacing
\usepackage{enumitem}

% Basic list styling
\setlist[itemize,1]{
    leftmargin=18pt,
    itemsep=3pt plus 1pt minus 1pt,
    parsep=0pt,
    topsep=6pt plus 2pt minus 2pt,
    label=\color{accent}$\bullet$
}
\setlist[itemize,2]{
    leftmargin=16pt,
    itemsep=2pt,
    parsep=0pt,
    topsep=3pt,
    label=\color{secondary}$\circ$
}
\setlist[itemize,3]{
    leftmargin=14pt,
    itemsep=1pt,
    parsep=0pt,
    topsep=2pt,
    label=\color{textgray}{\scriptsize$\blacksquare$}
}

% Enhanced enumerate styling
\setlist[enumerate,1]{
    leftmargin=22pt,
    itemsep=3pt plus 1pt minus 1pt,
    parsep=0pt,
    topsep=6pt plus 2pt minus 2pt,
    label=\color{accent}\arabic*.
}
\setlist[enumerate,2]{
    leftmargin=20pt,
    itemsep=2pt,
    parsep=0pt,
    topsep=3pt,
    label=\color{secondary}(\alph*)
}
\setlist[enumerate,3]{
    leftmargin=18pt,
    itemsep=1pt,
    parsep=0pt,
    topsep=2pt,
    label=\color{textgray}\roman*.
}

% Description list styling
\setlist[description]{
    font=\normalfont\color{accent}\bfseries,
    labelwidth=2cm,
    leftmargin=2.5cm,
    itemsep=4pt,
    parsep=0pt
}

% Custom task lists with icons (requires fontawesome5)
\newlist{tasklist}{itemize}{1}
\setlist[tasklist]{
    leftmargin=20pt,
    label=\color{success}\faCheck,
    itemsep=4pt,
    parsep=2pt
}

\newlist{pendingtask}{itemize}{1}
\setlist[pendingtask]{
    leftmargin=20pt,
    label=\color{warning}\faHourglass,
    itemsep=4pt,
    parsep=2pt
}

\newlist{failedtask}{itemize}{1}
\setlist[failedtask]{
    leftmargin=20pt,
    label=\color{accent}\faTimes,
    itemsep=4pt,
    parsep=2pt
}

% Priority list environments
\newlist{highpriority}{itemize}{1}
\setlist[highpriority]{
    leftmargin=20pt,
    label=\color{accent}\faExclamationCircle,
    itemsep=4pt
}

\newlist{mediumpriority}{itemize}{1}
\setlist[mediumpriority]{
    leftmargin=20pt,
    label=\color{warning}\faExclamation,
    itemsep=4pt
}

\newlist{lowpriority}{itemize}{1}
\setlist[lowpriority]{
    leftmargin=20pt,
    label=\color{success}\faInfoCircle,
    itemsep=4pt
}

% TikZ bullets for professional styling
\newcommand{\tikzbullet}[1]{%
    \tikz[baseline=-0.7ex]\node[circle,fill=#1,inner sep=1.5pt]{};%
}

% Compact and expanded list variants
\newenvironment{compactlist}
    {\begin{itemize}[nosep]}
    {\end{itemize}}

\newenvironment{expandedlist}
    {\begin{itemize}[itemsep=8pt,parsep=4pt]}
    {\end{itemize}}

% ========== TABLES ==========
% Instructions: Packages for professional table formatting
\usepackage{booktabs}
\usepackage{tabularx}
\usepackage{caption}
\usepackage{longtable} % Multi-page tables
\usepackage{array} % Enhanced column types
\usepackage{multirow} % Spanning multiple rows in tables

% Define new column types for better table formatting
\newcolumntype{L}[1]{>{\raggedright\arraybackslash}p{#1}}
\newcolumntype{C}[1]{>{\centering\arraybackslash}p{#1}}
\newcolumntype{R}[1]{>{\raggedleft\arraybackslash}p{#1}}
\newcolumntype{N}{>{\raggedleft\arraybackslash}X} % Right-aligned number column

% Table styling
\captionsetup{font=small, labelfont=bf, labelsep=colon}

% Additional table colors
\definecolor{cardbg}{HTML}{F7F9FC}
\definecolor{headerbg}{HTML}{4A90E2}
\definecolor{headertext}{HTML}{FFFFFF}
\definecolor{rowalt}{HTML}{F1F4F9}
\definecolor{frameborder}{HTML}{D3DCE6}

% ========== CODE LISTINGS ==========
% Instructions: For code syntax highlighting
\usepackage{listings}
\lstset{
    basicstyle=\ttfamily\small,
    backgroundcolor=\color{codebg},
    commentstyle=\color{codecomment},
    keywordstyle=\color{codekey}\bfseries,
    stringstyle=\color{codestring},
    numbers=left,
    numberstyle=\tiny\color{textgray},
    numbersep=10pt,
    tabsize=4,
    breaklines=true,
    breakatwhitespace=false,
    frame=single,
    rulecolor=\color{border},
    framesep=8pt,
    showstringspaces=false
}

% Uncomment for advanced syntax highlighting with minted (requires Python)
\usepackage{minted}
\setminted{
    style=default,
    bgcolor=codebg,
    linenos=true,
    breaklines=true,
    tabsize=4,
    fontsize=\small,
    frame=single,
    framesep=8pt
}

% ========== ALGORITHMS ==========
% Instructions: For pseudocode or process descriptions
\usepackage[ruled,vlined]{algorithm2e}
\SetAlgoLined
\SetAlgoSkip{medskip}
\SetAlFnt{\footnotesize\sffamily}
\renewcommand{\algorithmcfname}{Algorithm}

% ========== ICON BOXES ==========
% Load fontawesome and awesomebox for icon support
\usepackage{fontawesome5}
\usepackage{awesomebox}

% ========== ADVANCED IMAGE HANDLING ==========
% Instructions: Professional image formatting with shapes and fallbacks

% Required packages (graphicx should already be loaded)
\usepackage[export,graphics,noclipbox]{adjustbox} % Load without redefining \clipbox command
\usepackage{varwidth}

% Create a box to use as a fallback for missing images
\newsavebox{\imagefallbackbox}

% Command for images with fallback text
% Usage: \imageWithFallback[options]{path}{width}{alt-text}
\newcommand{\imageWithFallback}[4][]{%
  \begingroup
  \sbox{\imagefallbackbox}{%
    \begin{varwidth}{\textwidth}
      \centering
      \textcolor{textgray}{\textbf{#4}}
    \end{varwidth}
  }%
  \IfFileExists{#2}{%
    \includegraphics[#1,width=#3]{#2}%
  }{%
    \fcolorbox{border}{background}{%
      \begin{minipage}[c][0.75\ht\imagefallbackbox][c]{#3}
        \centering
        \textcolor{textgray}{\faImage}\\[0.3em]
        \usebox{\imagefallbackbox}
      \end{minipage}%
    }%
  }%
  \endgroup
}

% Rectangular image with optional border and caption
% Usage: \rectImage[options]{path}{width}{caption}{alt-text}
\newcommand{\rectImage}[5][]{%
  \begin{figure}[htbp]
    \centering
    \fbox{\imageWithFallback[#1]{#2}{#3}{#5}}
    \ifx&#4&%
    \else
      \caption{#4}
    \fi
  \end{figure}
}

% Rounded corner image
% Usage: \roundedImage[options]{path}{width}{radius}{caption}{alt-text}
\newcommand{\roundedImage}[6][]{%
  \begin{figure}[htbp]
    \centering
    \begin{tikzpicture}
      \node[inner sep=0pt] (image) {%
        \imageWithFallback[#1]{#2}{#3}{#6}%
      };
      \begin{scope}
        \clip[rounded corners=#4] (image.south west) rectangle (image.north east);
        \node[inner sep=0pt] {\imageWithFallback[#1]{#2}{#3}{#6}};
      \end{scope}
    \end{tikzpicture}
    \ifx&#5&%
    \else
      \caption{#5}
    \fi
  \end{figure}
}

% Circular image
% Usage: \circularImage[options]{path}{diameter}{caption}{alt-text}
\newcommand{\circularImage}[5][]{%
  \begin{figure}[htbp]
    \centering
    \begin{tikzpicture}
      \node[inner sep=0pt] (image) {%
        \imageWithFallback[#1]{#2}{#3}{#5}%
      };
      \begin{scope}
        \clip (image.center) circle (#3/2);
        \node[inner sep=0pt] {\imageWithFallback[#1]{#2}{#3}{#5}};
      \end{scope}
    \end{tikzpicture}
    \ifx&#4&%
    \else
      \caption{#4}
    \fi
  \end{figure}
}

% Image with drop shadow
% Usage: \shadowImage[options]{path}{width}{caption}{alt-text}
\newcommand{\shadowImage}[5][]{%
  \begin{figure}[htbp]
    \centering
    \begin{tikzpicture}
      \node[inner sep=0pt, drop shadow={shadow xshift=2pt, shadow yshift=-2pt, opacity=0.5}] {%
        \imageWithFallback[#1]{#2}{#3}{#5}%
      };
    \end{tikzpicture}
    \ifx&#4&%
    \else
      \caption{#4}
    \fi
  \end{figure}
}

% Frame image with customizable style
% Usage: \framedImage[options]{path}{width}{frame-color}{caption}{alt-text}
\newcommand{\framedImage}[6][]{%
  \begin{figure}[htbp]
    \centering
    \begin{tikzpicture}
      \node[inner sep=0pt] (image) {%
        \imageWithFallback[#1]{#2}{#3}{#6}%
      };
      \draw[line width=2pt, color=#4] (image.south west) rectangle (image.north east);
    \end{tikzpicture}
    \ifx&#5&%
    \else
      \caption{#5}
    \fi
  \end{figure}
}

% Full-width responsive image
% Usage: \fullwidthImage[options]{path}{height}{caption}{alt-text}
\newcommand{\fullwidthImage}[5][]{%
  \begin{figure}[htbp]
    \centering
    \makebox[\textwidth][c]{%
      \imageWithFallback[#1,height=#3,width=\textwidth,keepaspectratio]{#2}{\textwidth}{#5}%
    }
    \ifx&#4&%
    \else
      \caption{#4}
    \fi
  \end{figure}
}

% Image with zoom effect on specific area
% Usage: \zoomedImage[options]{path}{width}{zoom-x}{zoom-y}{zoom-width}{zoom-factor}{caption}{alt-text}
\newcommand{\zoomedImage}[9][]{%
  \begin{figure}[htbp]
    \centering
    \begin{tikzpicture}
      \node[inner sep=0pt] (image) {%
        \imageWithFallback[#1]{#2}{#3}{#9}%
      };
      \draw[red, thick] (#4,#5) rectangle ++(#6,#6);
      \begin{scope}[shift={(5cm,0)}]
        \draw[thick] (0,0) circle (1.5cm);
        \clip (0,0) circle (1.5cm);
        \node[inner sep=0pt] at (0,0) {%
          \imageWithFallback[#1]{#2}{#3*#7}{#9}%
        };
      \end{scope}
    \end{tikzpicture}
    \ifx&#8&%
    \else
      \caption{#8}
    \fi
  \end{figure}
}

% Responsive image grid (2x2)
% Usage: \imageGrid{path1}{path2}{path3}{path4}{grid-width}{caption}{alt-text}
\newcommand{\imageGrid}[7]{%
  \begin{figure}[htbp]
    \centering
    \begin{tabular}{cc}
      \imageWithFallback[width=#5/2-4pt]{#1}{#5/2-4pt}{#7} &
      \imageWithFallback[width=#5/2-4pt]{#2}{#5/2-4pt}{#7} \\
      \imageWithFallback[width=#5/2-4pt]{#3}{#5/2-4pt}{#7} &
      \imageWithFallback[width=#5/2-4pt]{#4}{#5/2-4pt}{#7} \\
    \end{tabular}
    \ifx&#6&%
    \else
      \caption{#6}
    \fi
  \end{figure}
}

% ========== TITLE PAGE ==========
% Instructions: Controls the appearance of the title page
\usepackage{titling}
\pretitle{
    \begin{center}
    \vspace{0.5in}
    \color{primary}\Huge\bfseries\sffamily
}
\posttitle{
    \end{center}
    \vspace{0.3in}
    \begin{center}
    \color{border}\rule{0.6\textwidth}{2pt}
    \end{center}
    \vspace{0.2in}
}
\preauthor{\begin{center}\color{secondary}\large\sffamily}
\postauthor{\end{center}}
\predate{\begin{center}\color{textgray}\large\sffamily}
\postdate{\end{center}\vspace{0.5in}}

% ========== DOCUMENT METADATA ==========
% Instructions: Replace with your actual document information
\title{Your Main Document Title Here\\[0.3em]Optional Subtitle Goes Here}
\author{Author Name or Organization}
\date{\today} % This will show current date, or replace with specific date 

% ========== DOCUMENT BEGIN ==========
\begin{document}

% ========== COVER PAGE (OPTIONAL) ==========
% Include selected cover if one is specified
\ifthenelse{\equal{\covertype}{none}}{
    % No cover page selected, continue with regular title page
}{
    % Create cover page based on selected cover type
    \clearpage
    \thispagestyle{empty}
    
    % Define placeholder commands if not already defined
    \providecommand{\TitlePlaceholder}{\@title}
    \providecommand{\SubtitlePlaceholder}{Professional documentation system}
    \providecommand{\SummaryTitlePlaceholder}{Executive Summary}
    \providecommand{\SummaryTextPlaceholder}{This document provides comprehensive information on our latest research and development efforts.}
    \providecommand{\PlatformsList}{Web, Mobile, Desktop, Cloud}
    \providecommand{\FocusList}{Performance, Scalability, Security, Usability}
    \providecommand{\FocusItemOne}{Technical Excellence}
    \providecommand{\FocusItemTwo}{User Experience}
    \providecommand{\FocusItemThree}{Scalability}
    \providecommand{\FocusItemFour}{Market Advantage}
    \providecommand{\AuthorPlaceholder}{\@author}
    \providecommand{\DatePlaceholder}{\@date}
    \providecommand{\ClassificationPlaceholder}{\documentClassification}
    \providecommand{\DistributionPlaceholder}{All Departments}
    \providecommand{\VersionPlaceholder}{\documentVersion}
    \providecommand{\StatusPlaceholder}{\documentStatus}
    \providecommand{\ClassificationLevelPlaceholder}{Confidential}
    
    % Directly include the selected cover file
    \input{covers/cover\covertype.tex}
    
    \newpage
}

% ========== TITLE PAGE ==========
% Instructions: This creates the title page automatically
\maketitle
\thispagestyle{empty}

% ========== EXECUTIVE SUMMARY BOX ==========
% Instructions: Replace with your document summary or overview
\begin{infobox}
\textbf{From \brandname{\companyName}}: This document demonstrates the advanced typesetting capabilities of our \productname{Lagen} documentation system. It showcases professional formatting features including multi-page tables, enhanced column types, advanced mathematical typesetting, and sophisticated branding controls.
\end{infobox}

\vfill
% ========== DOCUMENT CLASSIFICATION ==========
% Instructions: Replace with your document's classification information
\begin{center}
\color{textgray}\footnotesize\sffamily
\textbf{Document Classification:} \documentClassification\\
\textbf{Distribution:} All Departments\\
\textbf{Version:} \documentVersion \quad \textbf{Pages:} \pageref{LastPage}
\end{center}

\newpage

% ========== TABLE OF CONTENTS ==========
% Instructions: This generates automatically based on your sections
\tableofcontents
\newpage

% ========== MAIN CONTENT SECTIONS ==========

\section{Content Demonstration}
\label{sec:demonstration}
% Instructions: Replace with your first major section title

\lettrine[lines=3, loversize=0.1]{T}{his} demonstrates a drop cap at the beginning of a section using the lettrine package. It adds an elegant typographic touch to your document. The rest of this paragraph shows how the text wraps around the large initial letter.

This document now features \nameref{sec:tables} spanning multiple pages, proper formatting of units like \SI{42}{\kilo\watt\hour} or \SI{1.25}{\USD}, and advanced mathematical typesetting with $\displaystyle\int_{0}^{\infty} e^{-x^2} dx = \frac{\sqrt{\pi}}{2}$.

\subsection{Advanced Typography Features}

\needspace{4\baselineskip}
Our branding is consistent with \brandname{\companyName} and \productname{Lagen Documentation System} appearing throughout the document. 

Here we demonstrate quotation handling with csquotes: "This is properly formatted as a quotation with smart quotes." Notice how the quotes are properly curled rather than using straight quotes.

\begin{displayquote}
This is a block quotation that shows proper indentation and formatting. It's useful for longer quotes that deserve special emphasis or visual distinction from the main text.
\end{displayquote}

\subsection{Cross-References and Citations}

This section demonstrates intelligent cross-referencing. You can refer to Section~\ref{sec:boxes} for box demonstrations, or Section~\nameref{sec:tables} for table examples.

\needspace{4\baselineskip}
For bibliography management, you can cite sources like \texttt{\textbackslash cite\{example\}} and have them automatically formatted based on your chosen citation style.\footnote{Requires a .bib file and proper setup with biblatex and biber.}

\subsection{Subsection Example}
% Instructions: Use subsections to organize content within main sections

This demonstrates how to organize content using subsections.

\begin{itemize}
    \item This is a first-level bullet point
    \item This is another bullet point with nested items:
    \begin{itemize}
        \item This is a second-level bullet point
        \item This is another second-level bullet point
    \end{itemize}
\end{itemize}

\subsubsection{Nested Subsection}
% Instructions: Use subsubsections for further detail within subsections

This shows a third-level heading for detailed organization.

\section{Table Examples}
\label{sec:tables}
% Instructions: This section demonstrates modern table formatting

This section shows how to create professional-looking tables with color and styling, now with multi-page support and enhanced column types.

\subsection{Enhanced Table Examples}

The following demonstrates a table with custom column types and multirow capabilities:

\begin{table}[htbp]
\caption{Advanced Table Formatting}
\centering
\rowcolors{2}{rowalt}{white}
\begin{tabularx}{\textwidth}{L{2.5cm}C{4cm}R{2cm}>{\raggedleft\arraybackslash}X}
\rowcolor{headerbg}
\textcolor{headertext}{\textbf{Category}} &
\textcolor{headertext}{\textbf{Description}} &
\textcolor{headertext}{\textbf{Value}} &
\textcolor{headertext}{\textbf{Progress}} \\
\toprule
\multirow{2}{*}{Development} & Initial framework design & \SI{15000}{\USD} & 95\% \\
 & Component integration & \SI{8500}{\USD} & 45\% \\
\midrule
\multirow{3}{*}{Testing} & Unit testing framework & \SI{4200}{\USD} & 100\% \\
 & Integration testing & \SI{6300}{\USD} & 60\% \\
 & Performance testing & \SI{5500}{\USD} & 30\% \\
\midrule
Deployment & Cloud infrastructure & \SI{12000}{\USD} & 15\% \\
\bottomrule
\end{tabularx}
\end{table}

\subsection{Multi-page Table Example}

Here's an example of a table that can span multiple pages:

\begin{longtable}{p{2cm}p{4cm}p{3cm}p{4cm}}
\caption{Multi-page Project Timeline and Milestones} \\
\rowcolor{headerbg}
\textcolor{headertext}{\textbf{Date}} &
\textcolor{headertext}{\textbf{Milestone}} &
\textcolor{headertext}{\textbf{Responsible}} &
\textcolor{headertext}{\textbf{Deliverables}} \\
\toprule
\endfirsthead

\multicolumn{4}{c}{\tablename\ \thetable{} -- continued from previous page} \\
\rowcolor{headerbg}
\textcolor{headertext}{\textbf{Date}} &
\textcolor{headertext}{\textbf{Milestone}} &
\textcolor{headertext}{\textbf{Responsible}} &
\textcolor{headertext}{\textbf{Deliverables}} \\
\toprule
\endhead

\midrule \multicolumn{4}{r}{Continued on next page} \\
\endfoot

\bottomrule
\endlastfoot

2023-01-15 & Project Initiation & Project Manager & Project charter, stakeholder register \\
\rowcolor{rowalt}
2023-02-01 & Requirements Gathering & Business Analyst & Requirements document, user stories \\
2023-02-15 & System Architecture & Solution Architect & Architecture document, technology stack \\
\rowcolor{rowalt}
2023-03-01 & Design Phase Completion & Design Team & UI/UX designs, wireframes \\
2023-03-15 & Development Sprint 1 & Development Team & Core functionality implementation \\
\rowcolor{rowalt}
2023-04-01 & Development Sprint 2 & Development Team & Feature set 1 completion \\
2023-04-15 & Development Sprint 3 & Development Team & Feature set 2 completion \\
\rowcolor{rowalt}
2023-05-01 & Alpha Testing & QA Team & Test reports, bug tracking \\
2023-05-15 & Beta Release & Release Manager & Beta deployment, user feedback \\
\rowcolor{rowalt}
2023-06-01 & User Acceptance Testing & QA Team & UAT report, sign-off documents \\
2023-06-15 & Performance Optimization & Performance Engineer & Optimization report, benchmarks \\
\rowcolor{rowalt}
2023-07-01 & Security Audit & Security Team & Security assessment, vulnerability report \\
2023-07-15 & Documentation & Technical Writer & User manual, API documentation \\
\rowcolor{rowalt}
2023-08-01 & Final QA & QA Team & Final test report, release readiness \\
2023-08-15 & Production Deployment & DevOps Team & Deployment checklist, monitoring setup \\
\rowcolor{rowalt}
2023-09-01 & Post-Launch Support & Support Team & Support plan, escalation procedures \\
2023-09-15 & Project Review & Project Manager & Lessons learned, project closure report \\
\end{longtable}

\subsection{Data Analysis Table}

\begin{tcolorbox}[colback=cardbg, colframe=frameborder, coltitle=white,
    fonttitle=\bfseries, title=📊 Data Analysis Results,
    enhanced, boxrule=0.5pt, arc=4pt, left=10pt, right=10pt, top=8pt, bottom=8pt,
    attach boxed title to top left={xshift=10pt, yshift=-2mm},
    boxed title style={colback=accent, arc=2pt}
]

\rowcolors{2}{rowalt}{white}

\begin{tabularx}{\linewidth}{>{\raggedright\arraybackslash\bfseries}X >{\raggedright\arraybackslash}X c c}
    \rowcolor{headerbg}
    \textcolor{headertext}{Category} &
    \textcolor{headertext}{Description} &
    \textcolor{headertext}{Value} &
    \textcolor{headertext}{Status} \\
    \toprule
    Project A & Implementation phase & 95\% & Complete  \\
    Project B & Analysis phase & 45\% & In progress \\
    Project C & Planning phase & 20\% & Started \\
    Project D & Testing phase & 80\% & Advanced \\
    Project E & Deployment phase & 10\% & Initial \\
    \bottomrule
\end{tabularx}

\end{tcolorbox}

\section{Mathematical Examples}
\label{sec:math}

\subsection{Equation Examples}

The enhanced math support allows for beautifully typeset equations:

\begin{align}
E &= mc^2 \\
F &= G\frac{m_1 m_2}{r^2} \\
\nabla \times \vec{E} &= -\frac{\partial \vec{B}}{\partial t}
\end{align}

\subsection{Units and Numbers}

With the siunitx package, we can consistently format units:

\begin{itemize}
    \item The battery capacity is \SI{5000}{\milli\ampere\hour}
    \item The system requires \SI{240}{\volt} power
    \item The project will take approximately \SI{3.5}{\year} to complete
    \item The budget is \SI{2500000}{\USD} for the first phase
    \item Normal room temperature is \SI{20}{\celsius}
    \item The server processes \SI{5.2}{\giga\byte} of data per minute
\end{itemize}

\section{Pseudocode and Algorithms}
\label{sec:algorithms}

\subsection{Basic Algorithm Example}

Here is a simple algorithm example using the algorithm2e package:

\begin{algorithm}[H]
	\SetAlgoLined
	\KwData{Input data $X$, parameters $\alpha$, $\beta$}
	\KwResult{Optimized output $Y$}
	Initialize $Y \gets \emptyset$\;
	\For{each element $x \in X$}{
		$y \gets \text{Process}(x, \alpha)$\;
		\If{$y > \beta$}{
			$Y \gets Y \cup \{y\}$\;
		}
		\Else{
			$y \gets \text{AlternativeProcess}(x)$\;
			$Y \gets Y \cup \{y\}$\;
		}
	}
	\Return{$Y$}\;
	\caption{Basic Optimization Algorithm}
\end{algorithm}

\section{Advanced Box Types}
\label{sec:boxes}
% Instructions: Examples of how to use the three types of callout boxes

Here are examples of different types of callout boxes:

\begin{successbox}
\textbf{Success Box:} Use this for highlighting positive outcomes or important accomplishments.
\end{successbox}

\begin{alertbox}
\textbf{Alert Box:} Use this for important cautions or critical information.
\end{alertbox}

\begin{infobox}[title=Information Box with Title]
This demonstrates an information box with a title. Use for supplementary information.
\end{infobox}

\subsection{Icon-Enhanced Boxes}

The awesomebox package provides boxes with built-in icons:

\awesomebox[violet]{2pt}{\faInfo}{violet}{This is a note with an icon. It's useful for additional information that deserves attention.}

\awesomebox[teal]{2pt}{\faLightbulb}{teal}{This is a tip with an icon. It provides helpful suggestions or shortcuts.}

\awesomebox[orange]{2pt}{\faExclamationTriangle}{orange}{This is a warning with an icon. It alerts readers to potential issues or mistakes.}

\awesomebox[red]{2pt}{\faExclamationCircle}{red}{This is a caution with an icon. It emphasizes important concerns.}

\awesomebox[blue]{2pt}{\faInfoCircle}{blue}{This is an important box with an icon. It highlights critical information.}

\subsection{Logo-Enhanced Boxes}

The bclogo package offers boxes with logos:

\bclogo[logo=\bcattention]{This box has an attention logo to alert readers to something important.}

\bclogo[logo=\bcquestion]{This box has a question logo for FAQs or exploration questions.}

\bclogo[logo=\bccrayon]{This box has a crayon logo that might be used for creative exercises or notes.}

\section{Code Listings and Algorithms}

\subsection{Code Snippets with Syntax Highlighting}

Here's an example of code with syntax highlighting using the listings package:

\begin{lstlisting}[language=Python, caption=Sample Python Code]
# This is a comment
def hello_world():
    """Docstring for function"""
    print("Hello, world!")
    return True

# Call the function
result = hello_world()
if result:
    print("Function executed successfully")
\end{lstlisting}

% Uncomment this if you have minted package configured
\begin{minted}[frame=single, framesep=8pt]{javascript}
// This is a comment
function helloWorld() {
  // Print a message
  console.log("Hello, world!");
  return true;
}

// Call the function
const result = helloWorld();
if (result) {
  console.log("Function executed successfully");
}
\end{minted}

\subsection{Algorithm Description}

Here's an example of an algorithm description using the algorithm2e package:

\begin{algorithm}[H]
\caption{Sample Algorithm}
\SetAlgoLined
\KwData{Input data $X$}
\KwResult{Output result $Y$}
Initialize parameters\;
\For{each element in $X$}{
    Process the element\;
    \If{condition is met}{
        Perform special operation\;
    }
    Update intermediate results\;
}
Calculate final result $Y$\;
\Return{$Y$}\;
\end{algorithm}

\section{List Examples}
% Instructions: Examples of different list types

\subsection{Bullet Point Lists}
% Instructions: Standard bullet points for unordered information

\begin{itemize}
    \item \textbf{First point:} Explanation with emphasized beginning.
    
    \item \textbf{Second point:} Another example with supporting detail.
\end{itemize}

\subsection{Numbered Lists}
% Instructions: Use enumerate environment for ordered/sequential information

\begin{enumerate}
    \item \textbf{First step:} Explanation of the first step in a process.
    
    \item \textbf{Second step:} Details about the second step or item.
\end{enumerate}

\section{Analysis Example}

This section shows how to structure content with multiple levels.

\begin{infobox}[title=Key Points]
\textbf{Point 1:} First key point\\
\textbf{Point 2:} Second key point\\
\textbf{Point 3:} Third key point
\end{infobox}

\section{References}
% Instructions: Section for citations and resources

% Example bibliography entries for demonstration
\begin{filecontents*}{example.bib}
@article{example2023,
  author = {Smith, John},
  title = {Example Title},
  year = {2023},
  journal = {Journal of Examples},
  volume = {10},
  number = {2},
  pages = {100--120}
}
\end{filecontents*}
% Note: In a real document, you would place biblatex entries in a separate .bib file

\textbf{References:}
\begin{itemize}
    \item \href{https://example.com/source1}{Reference 1 - Brief Description}
    \item \href{https://example.com/source2}{Reference 2 - Brief Description}
\end{itemize}

% Uncomment to print bibliography when you have a .bib file
% \printbibliography[heading=bibintoc, title={Bibliography}]

% Instructions: This label is used for page counting in the footer
\label{LastPage}
\end{document}
