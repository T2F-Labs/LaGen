% ========== ENHANCED TABLE OF CONTENTS ==========
% Comprehensive navigation system with all lists and enhanced formatting
% Includes TOC, figures, tables, algorithms, code listings, and boxes

% Add Symphony branding header
\begin{center}
\includegraphics[width=0.3\textwidth]{\SymphonyLogoPath}\\[0.5cm]
{\Large\textbf{SYMPHONY DOCUMENTATION NAVIGATION}}\\[0.3cm]
\textit{Comprehensive Guide to AI-First Development Environment}
\end{center}

\vspace{1cm}

% Table of Contents with enhanced formatting
\renewcommand{\contentsname}{Table of Contents}
\tableofcontents
\clearpage

% List of Figures with description
\section*{Visual Elements Guide}
\addcontentsline{toc}{section}{Visual Elements Guide}

\begin{infobox}[title=Navigation Guide for Visual Elements]
This section provides comprehensive navigation for all visual elements in the Symphony documentation, including figures, tables, algorithms, code examples, and information boxes used throughout the document.
\end{infobox}

\subsection*{List of Figures}
\listoffigures

\subsection*{List of Tables}
\listoftables

% List of Algorithms (if any algorithms are present)
\ifthenelse{\equal{\EnableAlgorithms}{true}}{
\subsection*{List of Algorithms}
\listofalgorithms
}{}

% List of Code Listings (if code module is enabled)
\ifthenelse{\equal{\EnableCode}{true}}{
\subsection*{List of Code Listings}
\lstlistoflistings
}{}

\subsection*{Information Boxes Reference}
The following types of information boxes are used throughout this documentation:

\begin{description}[leftmargin=3cm,labelwidth=2.5cm]
    \item[\textbf{Info Boxes}] \colorbox{blue!20}{General information and explanations}
    \item[\textbf{Success Boxes}] \colorbox{green!20}{Achievements and positive outcomes}
    \item[\textbf{Alert Boxes}] \colorbox{orange!20}{Important warnings and considerations}
    \item[\textbf{Technical Boxes}] \colorbox{purple!20}{Technical specifications and details}
\end{description}

\section*{Reader's Navigation Guide}

\begin{successbox}
\textbf{For Different Reader Types:}
\begin{compactlist}
    \item \textbf{Researchers \& Academics}: Focus on Chapters 1-3, 9-10, 13-16, 24-26
    \item \textbf{Developers}: Emphasize Chapters 4, 7-8, 11, 19-21, Appendices C-F
    \item \textbf{Architects}: Concentrate on Chapters 5-6, 12, 17-18, Appendix B
    \item \textbf{Users}: Begin with Chapters 1-3, 9, 20, Appendices A \& D
\end{compactlist}
\end{successbox}

\section*{Document Structure Overview}

\begin{table}[h]
\centering
\begin{tabular}{@{}lll@{}}
\toprule
\textbf{Part} & \textbf{Chapters} & \textbf{Focus Area} \\
\midrule
Front Matter & i-viii & Introduction \& Navigation \\
Part I & 1-3 & Introduction \& Foundations \\
Part II & 4-5 & Technical Foundation \& Architecture \\
Part III & 6-8 & Core Infrastructure \& Microkernel \\
Part IV & 9-12 & AIDE, Intelligence \& Execution \\
Part V & 13-16 & Orchestration \& AI Models \\
Part VI & 17-18 & Data Management \& Storage \\
Part VII & 19-23 & Implementation \& Engineering \\
Part VIII & 24-26 & Evaluation \& Future Work \\
Appendices & A-G & Reference \& Supporting Material \\
\bottomrule
\end{tabular}
\caption{Document Structure Overview}
\end{table}

\clearpage