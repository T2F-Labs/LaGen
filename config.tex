% ========== FONT AND ENCODING ==========
% Instructions: These packages handle font setup and character encoding
% Keep these as-is unless you need different fonts
\usepackage{inputenc}
\usepackage[T1]{fontenc}
\usepackage{lmodern}
\usepackage{fontspec}
\usepackage{setspace}
\usepackage{xcolor}
\usepackage{morefloats}
\usepackage[table]{xcolor}
% Professional letter spacing for brand names
% \usepackage{letterspace} % Removed as fontspec will handle letterspacing
\usepackage{microtype}
% Dynamic font scaling for brand consistency
\usepackage{scalefnt}
% Replace "Inter" with your preferred font family if needed
\setmainfont{Inter}[
    UprightFont = *-Regular,
    BoldFont = *-SemiBold,
    ItalicFont = *-Italic,
    BoldItalicFont = *-SemiBoldItalic,
    LetterSpace = 3.0 % Base letter spacing for the font
]
\setsansfont{Inter}[
    UprightFont = *-Regular,
    BoldFont = *-SemiBold,
    ItalicFont = *-Italic,
    BoldItalicFont = *-SemiBoldItalic,
    LetterSpace = 3.0 % Base letter spacing for the font
]

% Define branded text macros with letterspacing using fontspec
\newcommand{\brandname}[1]{{\addfontfeatures{LetterSpace=12.0}\textsc{#1}}}
\newcommand{\productname}[1]{{\addfontfeatures{LetterSpace=8.0}\textbf{#1}}}

% ========== PAGE LAYOUT ==========
% Instructions: Adjust measurements to change margins and spacing
\usepackage{geometry}
\geometry{
    top=1.2in,
    bottom=1.2in,
    left=1.1in,
    right=1.1in,
    headheight=14pt,
    headsep=0.3in,
    footskip=0.4in
}

% Better float placement control
\usepackage{placeins}
% Ensure floats appear after their references
\usepackage{flafter}
% Bug fixes for LaTeX kernel
\usepackage{fixltx2e}

% ========== COLORS ==========
% Instructions: Modify these color definitions to match your brand/theme
\definecolor{primary}{HTML}{0F172A}        % Main text color - change as needed
\definecolor{secondary}{HTML}{334155}      % Secondary text color
\definecolor{accent}{HTML}{2563EB}         % Links and highlights
\definecolor{lightaccent}{HTML}{3B82F6}    % Lighter accent variant
\definecolor{success}{HTML}{059669}        % Success/positive elements
\definecolor{warning}{HTML}{D97706}        % Warning/attention elements
\definecolor{background}{HTML}{F8FAFC}     % Light background for boxes
\definecolor{border}{HTML}{E2E8F0}        % Borders and lines
\definecolor{textgray}{HTML}{64748B}      % Muted text
\definecolor{lightgray}{HTML}{F1F5F9}     % Table alternating rows
\definecolor{codebg}{HTML}{F8F8F8}        % Code block background
\definecolor{codecomment}{HTML}{6A737D}   % Code comments
\definecolor{codestring}{HTML}{032F62}    % Code strings
\definecolor{codekey}{HTML}{D73A49}       % Code keywords

% ========== TYPOGRAPHY ==========
% Instructions: These settings control paragraph spacing and text optimization
\usepackage{microtype}
\usepackage{parskip}
\setlength{\parindent}{0pt}
\setlength{\parskip}{8pt plus 2pt minus 1pt}

% ========== ADVANCED TEXT FEATURES ==========
% Instructions: For elegant drop caps at section beginnings
\usepackage{lettrine}
% Configuration for lettrine (drop caps)
\renewcommand{\LettrineFontHook}{\color{accent}\bfseries}
\setcounter{DefaultLines}{3}
\renewcommand{\DefaultLoversize}{0.1}
\renewcommand{\DefaultLraise}{0}

% Instructions: For proper quotation handling and formatting
\usepackage[autostyle=true,german=quotes,english=american]{csquotes}
\MakeOuterQuote{"}

% Instructions: For better paragraph flow control
\usepackage[all]{nowidow}
\usepackage{needspace}

% ========== SECTION STYLING ==========
% Instructions: Customize section headings appearance here
\usepackage{titlesec}
\titleformat{\section}
    {\color{primary}\fontsize{16}{20}\bfseries\sffamily}
    {\color{accent}\thesection}
    {1em}
    {}
    [\vspace{2pt}\color{border}\hrule height 0.8pt\vspace{6pt}]

\titleformat{\subsection}
    {\color{secondary}\fontsize{14}{18}\bfseries\sffamily}
    {\color{accent}\thesubsection}
    {0.8em}
    {}
    [\vspace{4pt}]

\titleformat{\subsubsection}
    {\color{secondary}\fontsize{12}{16}\bfseries\sffamily}
    {\color{accent}\thesubsubsection}
    {0.6em}
    {}

\titlespacing*{\section}{0pt}{20pt plus 4pt minus 2pt}{8pt plus 2pt minus 2pt}
\titlespacing*{\subsection}{0pt}{16pt plus 3pt minus 2pt}{6pt plus 2pt minus 1pt}
\titlespacing*{\subsubsection}{0pt}{12pt plus 2pt minus 1pt}{4pt plus 1pt minus 1pt}

% ========== HEADERS AND FOOTERS ==========
% Instructions: Replace "Your Document Title" and footer text with your content
\usepackage{fancyhdr}
\usepackage{lastpage} % For "Page X of Y" formatting
% \usepackage{scrlayer-scrpage} % Removed to avoid conflicts with fancyhdr
\pagestyle{fancy}
\fancyhf{}
\renewcommand{\headrulewidth}{0.5pt}
\renewcommand{\footrulewidth}{0pt}
\fancyhead[L]{\color{textgray}\small\sffamily Your Document Title Here}
\fancyhead[R]{\color{textgray}\small\sffamily\thepage}
\fancyfoot[C]{\color{textgray}\footnotesize\sffamily Page \thepage{} of \pageref{LastPage} | Document Footer Text Here}

% ========== TABLE OF CONTENTS ==========
% Instructions: These settings style the table of contents
\usepackage{tocloft}
\renewcommand{\cftsecleader}{\cftdotfill{\cftdotsep}}
\renewcommand{\cftsubsecleader}{\cftdotfill{\cftdotsep}}
\setlength{\cftbeforesecskip}{6pt}
\setlength{\cftbeforesubsecskip}{3pt}
\renewcommand{\cftsecfont}{\color{primary}\bfseries\sffamily}
\renewcommand{\cftsubsecfont}{\color{secondary}\sffamily}
\renewcommand{\cftsecpagefont}{\color{accent}\bfseries\sffamily}
\renewcommand{\cftsubsecpagefont}{\color{accent}\sffamily}

% ========== ADVANCED REFERENCE MANAGEMENT ==========
% Instructions: For sophisticated bibliography management
\usepackage[
    backend=biber,
    style=authoryear,
    sorting=nyt,
    maxbibnames=99,
    giveninits=true
]{biblatex}
% If you have a bibliography file, uncomment and specify it here:
% \addbibresource{references.bib}

% Instructions: For page-aware references and cross-references
% IMPORTANT: the order matters! hyperref first, then nameref, then varioref, and cleveref LAST
\usepackage{hyperref}
\hypersetup{
    colorlinks=true,
    linkcolor=accent,
    urlcolor=lightaccent,
    citecolor=accent,
    filecolor=accent,
    bookmarks=true,
    bookmarksopen=true,
    bookmarksnumbered=true,
    pdfstartview=FitH,
    pdfpagelayout=SinglePage,
    pdftitle={Your PDF Title Here},
    pdfauthor={Your Name Here},
    pdfsubject={Document Subject Here},
    pdfkeywords={keyword1, keyword2, keyword3}
}
\usepackage{nameref} % Reference section names, not just numbers
\usepackage{varioref} % For page-aware references
\labelformat{figure}{Figure~#1}
\labelformat{table}{Table~#1}
\labelformat{equation}{Equation~#1}
\labelformat{section}{Section~#1}

% ========== MATHEMATICAL TYPESETTING ==========
% Essential enhanced math environments
\usepackage{amsmath}
% Extended mathematical symbol collection
\usepackage{amssymb}
% Fix common mathematical typography issues
\usepackage{fixmath}
% Proper formatting of numbers and units
\usepackage{siunitx}

% Configure siunitx for consistent number formatting
\sisetup{
    group-separator={,},
    group-minimum-digits=4,
    detect-weight=true,
    detect-family=true
}

% Define common units and constants
\DeclareSIUnit\USD{\$}
\DeclareSIUnit\hour{h}
\DeclareSIUnit\year{yr}

% ========== ENHANCED MATHEMATICAL TYPESETTING ==========
% Advanced math tools and additional environments
\usepackage{mathtools}
% Enhanced equation highlighting and boxes
\usepackage{empheq}
% Advanced cases environments
\usepackage{cases}
% Theorem environments for mathematical statements
\usepackage{amsthm}
% Advanced symbol sets
\usepackage{stmaryrd}
\usepackage{textcomp}
% Physics notation package for differential operators, brackets, etc.
\usepackage{physics}
% Tensor notation
\usepackage{tensor}
% Quantum mechanics notation
\usepackage{braket}
% Cross out terms in equations
\usepackage{cancel}

% Theorem environments setup
\theoremstyle{definition}
\newtheorem{definition}{Definition}[section]
\newtheorem{theorem}{Theorem}[section]
\newtheorem{lemma}[theorem]{Lemma}
\newtheorem{corollary}[theorem]{Corollary}
\newtheorem{proposition}[theorem]{Proposition}
\newtheorem{example}{Example}[section]

\theoremstyle{remark}
\newtheorem{remark}{Remark}[section]
\newtheorem{note}{Note}[section]

% Custom colors for math highlighting
\definecolor{mathblue}{RGB}{0,82,155}
\definecolor{mathred}{RGB}{204,0,0}
\definecolor{mathgreen}{RGB}{0,128,0}
\definecolor{mathpurple}{RGB}{128,0,128}

% Common mathematical sets
\newcommand{\R}{\mathbb{R}}
\newcommand{\C}{\mathbb{C}}
\newcommand{\N}{\mathbb{N}}
\newcommand{\Z}{\mathbb{Z}}
\newcommand{\Q}{\mathbb{Q}}
\newcommand{\F}{\mathbb{F}}

% Enhanced derivatives and differential operators
\renewcommand{\d}{\mathrm{d}}
\newcommand{\dt}{\frac{\d}{\d t}}
\newcommand{\dx}{\frac{\d}{\d x}}
\newcommand{\dy}{\frac{\d}{\d y}}
\newcommand{\dz}{\frac{\d}{\d z}}
\newcommand{\prt}{\partial}
\newcommand{\pdx}[1]{\frac{\partial #1}{\partial x}}
\newcommand{\pdy}[1]{\frac{\partial #1}{\partial y}}
\newcommand{\pdz}[1]{\frac{\partial #1}{\partial z}}
\newcommand{\pdt}[1]{\frac{\partial #1}{\partial t}}

% Vector calculus operators
\providecommand{\grad}{\nabla}
\providecommand{\divg}{\nabla \cdot}
\providecommand{\curl}{\nabla \times}
\providecommand{\lapl}{\nabla^2}

% Enhanced formatting for vectors and matrices
\newcommand{\vect}[1]{\mathbf{#1}}
\newcommand{\mat}[1]{\mathbf{#1}}
\newcommand{\T}{\mathsf{T}} % Transpose

% Probability and statistics notation
\newcommand{\prob}[1]{\mathrm{P}\left(#1\right)}
\newcommand{\E}[1]{\mathrm{E}\left[#1\right]}
\renewcommand{\var}[1]{\mathrm{Var}\left(#1\right)}
\newcommand{\cov}[1]{\mathrm{Cov}\left(#1\right)}
\newcommand{\normal}{\mathcal{N}}
\newcommand{\uniform}{\mathcal{U}}

% Specialized math operators
\providecommand{\tr}{\operatorname{tr}}
\providecommand{\diag}{\operatorname{diag}}
\providecommand{\rank}{\operatorname{rank}}
\newcommand{\sign}{\operatorname{sign}}
\providecommand{\lcm}{\operatorname{lcm}}
\providecommand{\gcd}{\operatorname{gcd}}

% Box commands for important equations
\newcommand{\boxedeq}[1]{\begin{empheq}[box=\fbox]{equation}#1\end{empheq}}
\newcommand{\colorboxedeq}[2]{\begin{empheq}[box={\fcolorbox{#1}{white}}]{equation}#2\end{empheq}}

% ========== PAGE BREAK CONTROL ==========
% Instructions: For better page break control
\usepackage{afterpage}

% ========== BOXES AND CALLOUTS ==========
% Instructions: Required TikZ libraries for shadows in boxes
\usepackage{cleveref}
% Instructions: cleveref must be loaded AFTER hyperref, nameref, and varioref
\crefname{figure}{Figure}{Figures}
\crefname{table}{Table}{Tables}
\crefname{equation}{Equation}{Equations}
\crefname{section}{Section}{Sections}

\usepackage{tikz}
\usetikzlibrary{shadows}

% Instructions: Three types of callout boxes are defined below
\usepackage[most]{tcolorbox}

% Info box style - use for general information
\newtcolorbox{infobox}{
    colback=background,
    colframe=accent,
    coltext=primary,
    boxrule=1pt,
    arc=3pt,
    left=12pt,
    right=12pt,
    top=8pt,
    bottom=8pt,
    enhanced,
    drop shadow,
    fonttitle=\bfseries\sffamily\color{accent},
    title style={left color=accent!10, right color=accent!5},
    attach boxed title to top left={xshift=8pt, yshift=-2pt}
}

% Alert box style - use for cautions and important notes
\newtcolorbox{alertbox}{
    colback=warning!5,
    colframe=warning,
    coltext=primary,
    boxrule=1pt,
    arc=3pt,
    left=12pt,
    right=12pt,
    top=8pt,
    bottom=8pt,
    enhanced,
    drop shadow
}

% Success box style - use for positive highlights
\newtcolorbox{successbox}{
    colback=success!5,
    colframe=success,
    coltext=primary,
    boxrule=1pt,
    arc=3pt,
    left=12pt,
    right=12pt,
    top=8pt,
    bottom=8pt,
    enhanced,
    drop shadow
}

% ========== LOGO-ENHANCED BOXES ==========
\usepackage{bclogo}

% ========== LISTS ==========
% Instructions: These settings control list appearance and spacing
\usepackage{enumitem}

% Basic list styling
\setlist[itemize,1]{
    leftmargin=18pt,
    itemsep=3pt plus 1pt minus 1pt,
    parsep=0pt,
    topsep=6pt plus 2pt minus 2pt,
    label=\color{accent}$\bullet$
}
\setlist[itemize,2]{
    leftmargin=16pt,
    itemsep=2pt,
    parsep=0pt,
    topsep=3pt,
    label=\color{secondary}$\circ$
}
\setlist[itemize,3]{
    leftmargin=14pt,
    itemsep=1pt,
    parsep=0pt,
    topsep=2pt,
    label=\color{textgray}{\scriptsize$\blacksquare$}
}

% Enhanced enumerate styling
\setlist[enumerate,1]{
    leftmargin=22pt,
    itemsep=3pt plus 1pt minus 1pt,
    parsep=0pt,
    topsep=6pt plus 2pt minus 2pt,
    label=\color{accent}\arabic*.
}
\setlist[enumerate,2]{
    leftmargin=20pt,
    itemsep=2pt,
    parsep=0pt,
    topsep=3pt,
    label=\color{secondary}(\alph*)
}
\setlist[enumerate,3]{
    leftmargin=18pt,
    itemsep=1pt,
    parsep=0pt,
    topsep=2pt,
    label=\color{textgray}\roman*.
}

% Description list styling
\setlist[description]{
    font=\normalfont\color{accent}\bfseries,
    labelwidth=2cm,
    leftmargin=2.5cm,
    itemsep=4pt,
    parsep=0pt
}

% Custom task lists with icons (requires fontawesome5)
\newlist{tasklist}{itemize}{1}
\setlist[tasklist]{
    leftmargin=20pt,
    label=\color{success}\faCheck,
    itemsep=4pt,
    parsep=2pt
}

\newlist{pendingtask}{itemize}{1}
\setlist[pendingtask]{
    leftmargin=20pt,
    label=\color{warning}\faHourglass,
    itemsep=4pt,
    parsep=2pt
}

\newlist{failedtask}{itemize}{1}
\setlist[failedtask]{
    leftmargin=20pt,
    label=\color{accent}\faTimes,
    itemsep=4pt,
    parsep=2pt
}

% Priority list environments
\newlist{highpriority}{itemize}{1}
\setlist[highpriority]{
    leftmargin=20pt,
    label=\color{accent}\faExclamationCircle,
    itemsep=4pt
}

\newlist{mediumpriority}{itemize}{1}
\setlist[mediumpriority]{
    leftmargin=20pt,
    label=\color{warning}\faExclamation,
    itemsep=4pt
}

\newlist{lowpriority}{itemize}{1}
\setlist[lowpriority]{
    leftmargin=20pt,
    label=\color{success}\faInfoCircle,
    itemsep=4pt
}

% TikZ bullets for professional styling
\newcommand{\tikzbullet}[1]{%
    \tikz[baseline=-0.7ex]\node[circle,fill=#1,inner sep=1.5pt]{};%
}

% Compact and expanded list variants
\newenvironment{compactlist}
    {\begin{itemize}[nosep]}
    {\end{itemize}}

\newenvironment{expandedlist}
    {\begin{itemize}[itemsep=8pt,parsep=4pt]}
    {\end{itemize}}

% ========== TABLES ==========
% Instructions: Packages for professional table formatting
\usepackage{booktabs}
\usepackage{tabularx}
\usepackage{caption}
\usepackage{longtable} % Multi-page tables
\usepackage{array} % Enhanced column types
\usepackage{multirow} % Spanning multiple rows in tables

% Define new column types for better table formatting
\newcolumntype{L}[1]{>{\raggedright\arraybackslash}p{#1}}
\newcolumntype{C}[1]{>{\centering\arraybackslash}p{#1}}
\newcolumntype{R}[1]{>{\raggedleft\arraybackslash}p{#1}}
\newcolumntype{N}{>{\raggedleft\arraybackslash}X} % Right-aligned number column

% Table styling
\captionsetup{font=small, labelfont=bf, labelsep=colon}

% Additional table colors
\definecolor{cardbg}{HTML}{F7F9FC}
\definecolor{headerbg}{HTML}{4A90E2}
\definecolor{headertext}{HTML}{FFFFFF}
\definecolor{rowalt}{HTML}{F1F4F9}
\definecolor{frameborder}{HTML}{D3DCE6}

% ========== CODE LISTINGS ==========
% Instructions: For code syntax highlighting
\usepackage{listings}
\lstset{
    basicstyle=\ttfamily\small,
    backgroundcolor=\color{codebg},
    commentstyle=\color{codecomment},
    keywordstyle=\color{codekey}\bfseries,
    stringstyle=\color{codestring},
    numbers=left,
    numberstyle=\tiny\color{textgray},
    numbersep=10pt,
    tabsize=4,
    breaklines=true,
    breakatwhitespace=false,
    frame=single,
    rulecolor=\color{border},
    framesep=8pt,
    showstringspaces=false
}

% Uncomment for advanced syntax highlighting with minted (requires Python)
\usepackage{minted}
\setminted{
    style=default,
    bgcolor=codebg,
    linenos=true,
    breaklines=true,
    tabsize=4,
    fontsize=\small,
    frame=single,
    framesep=8pt
}

% ========== ALGORITHMS ==========
% Instructions: For pseudocode or process descriptions
\usepackage[ruled,vlined]{algorithm2e}
\SetAlgoLined
\SetAlgoSkip{medskip}
\SetAlFnt{\footnotesize\sffamily}
\renewcommand{\algorithmcfname}{Algorithm}

% ========== ICON BOXES ==========
% Load fontawesome and awesomebox for icon support
\usepackage{fontawesome5}
\usepackage{awesomebox}

% ========== ADVANCED IMAGE HANDLING ==========
% Instructions: Professional image formatting with shapes and fallbacks

% Required packages (graphicx should already be loaded)
\usepackage[export,graphics,graphicx]{adjustbox} % Load with options to avoid conflicts
\usepackage{varwidth}

% Create a box to use as a fallback for missing images
\newsavebox{\imagefallbackbox}

% Command for images with fallback text
% Usage: \imageWithFallback[options]{path}{width}{alt-text}
\newcommand{\imageWithFallback}[4][]{%
  \begingroup
  \sbox{\imagefallbackbox}{%
    \begin{varwidth}{\textwidth}
      \centering
      \textcolor{textgray}{\textbf{#4}}
    \end{varwidth}
  }%
  \IfFileExists{#2}{%
    \includegraphics[#1,width=#3]{#2}%
  }{%
    \fcolorbox{border}{background}{%
      \begin{minipage}[c][0.75\ht\imagefallbackbox][c]{#3}
        \centering
        \textcolor{textgray}{\faImage}\\[0.3em]
        \usebox{\imagefallbackbox}
      \end{minipage}%
    }%
  }%
  \endgroup
}

% Rectangular image with optional border and caption
% Usage: \rectImage[options]{path}{width}{caption}{alt-text}
\newcommand{\rectImage}[5][]{%
  \begin{figure}[htbp]
    \centering
    \fbox{\imageWithFallback[#1]{#2}{#3}{#5}}
    \ifx&#4&%
    \else
      \caption{#4}
    \fi
  \end{figure}
}

% Rounded corner image
% Usage: \roundedImage[options]{path}{width}{radius}{caption}{alt-text}
\newcommand{\roundedImage}[6][]{%
  \begin{figure}[htbp]
    \centering
    \begin{tikzpicture}
      \node[inner sep=0pt] (image) {%
        \imageWithFallback[#1]{#2}{#3}{#6}%
      };
      \begin{scope}
        \clip[rounded corners=#4] (image.south west) rectangle (image.north east);
        \node[inner sep=0pt] {\imageWithFallback[#1]{#2}{#3}{#6}};
      \end{scope}
    \end{tikzpicture}
    \ifx&#5&%
    \else
      \caption{#5}
    \fi
  \end{figure}
}

% Circular image
% Usage: \circularImage[options]{path}{diameter}{caption}{alt-text}
\newcommand{\circularImage}[5][]{%
  \begin{figure}[htbp]
    \centering
    \begin{tikzpicture}
      \node[inner sep=0pt] (image) {%
        \imageWithFallback[#1]{#2}{#3}{#5}%
      };
      \begin{scope}
        \clip (image.center) circle (#3/2);
        \node[inner sep=0pt] {\imageWithFallback[#1]{#2}{#3}{#5}};
      \end{scope}
    \end{tikzpicture}
    \ifx&#4&%
    \else
      \caption{#4}
    \fi
  \end{figure}
}

% Image with drop shadow
% Usage: \shadowImage[options]{path}{width}{caption}{alt-text}
\newcommand{\shadowImage}[5][]{%
  \begin{figure}[htbp]
    \centering
    \begin{tikzpicture}
      \node[inner sep=0pt, drop shadow={shadow xshift=2pt, shadow yshift=-2pt, opacity=0.5}] {%
        \imageWithFallback[#1]{#2}{#3}{#5}%
      };
    \end{tikzpicture}
    \ifx&#4&%
    \else
      \caption{#4}
    \fi
  \end{figure}
}

% Frame image with customizable style
% Usage: \framedImage[options]{path}{width}{frame-color}{caption}{alt-text}
\newcommand{\framedImage}[6][]{%
  \begin{figure}[htbp]
    \centering
    \begin{tikzpicture}
      \node[inner sep=0pt] (image) {%
        \imageWithFallback[#1]{#2}{#3}{#6}%
      };
      \draw[line width=2pt, color=#4] (image.south west) rectangle (image.north east);
    \end{tikzpicture}
    \ifx&#5&%
    \else
      \caption{#5}
    \fi
  \end{figure}
}

% Full-width responsive image
% Usage: \fullwidthImage[options]{path}{height}{caption}{alt-text}
\newcommand{\fullwidthImage}[5][]{%
  \begin{figure}[htbp]
    \centering
    \makebox[\textwidth][c]{%
      \imageWithFallback[#1,height=#3,width=\textwidth,keepaspectratio]{#2}{\textwidth}{#5}%
    }
    \ifx&#4&%
    \else
      \caption{#4}
    \fi
  \end{figure}
}

% Image with zoom effect on specific area
% Usage: \zoomedImage[options]{path}{width}{zoom-x}{zoom-y}{zoom-width}{zoom-factor}{caption}{alt-text}
\newcommand{\zoomedImage}[9][]{%
  \begin{figure}[htbp]
    \centering
    \begin{tikzpicture}
      \node[inner sep=0pt] (image) {%
        \imageWithFallback[#1]{#2}{#3}{#9}%
      };
      \draw[red, thick] (#4,#5) rectangle ++(#6,#6);
      \begin{scope}[shift={(5cm,0)}]
        \draw[thick] (0,0) circle (1.5cm);
        \clip (0,0) circle (1.5cm);
        \node[inner sep=0pt] at (0,0) {%
          \imageWithFallback[#1]{#2}{#3*#7}{#9}%
        };
      \end{scope}
    \end{tikzpicture}
    \ifx&#8&%
    \else
      \caption{#8}
    \fi
  \end{figure}
}

% Responsive image grid (2x2)
% Usage: \imageGrid{path1}{path2}{path3}{path4}{grid-width}{caption}{alt-text}
\newcommand{\imageGrid}[7]{%
  \begin{figure}[htbp]
    \centering
    \begin{tabular}{cc}
      \imageWithFallback[width=#5/2-4pt]{#1}{#5/2-4pt}{#7} &
      \imageWithFallback[width=#5/2-4pt]{#2}{#5/2-4pt}{#7} \\
      \imageWithFallback[width=#5/2-4pt]{#3}{#5/2-4pt}{#7} &
      \imageWithFallback[width=#5/2-4pt]{#4}{#5/2-4pt}{#7} \\
    \end{tabular}
    \ifx&#6&%
    \else
      \caption{#6}
    \fi
  \end{figure}
}

% ========== TITLE PAGE ==========
% Instructions: Controls the appearance of the title page
\usepackage{titling}
\pretitle{
    \begin{center}
    \vspace{0.5in}
    \color{primary}\Huge\bfseries\sffamily
}
\posttitle{
    \end{center}
    \vspace{0.3in}
    \begin{center}
    \color{border}\rule{0.6\textwidth}{2pt}
    \end{center}
    \vspace{0.2in}
}
\preauthor{\begin{center}\color{secondary}\large\sffamily}
\postauthor{\end{center}}
\predate{\begin{center}\color{textgray}\large\sffamily}
\postdate{\end{center}\vspace{0.5in}}

% ========== DOCUMENT METADATA ==========
% Instructions: Replace with your actual document information
\title{Your Main Document Title Here\\[0.3em]Optional Subtitle Goes Here}
\author{Author Name or Organization}
\date{\today} % This will show current date, or replace with specific date 