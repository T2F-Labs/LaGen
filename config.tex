% ========== FONT AND ENCODING ==========
% Instructions: These packages handle font setup and character encoding
% Keep these as-is unless you need different fonts
\usepackage{inputenc}
\usepackage[T1]{fontenc}
\usepackage{lmodern}
\usepackage{fontspec}
\usepackage{setspace}
\usepackage{xcolor}
\usepackage{morefloats}
\usepackage[table]{xcolor}
% Replace "Inter" with your preferred font family if needed
\setmainfont{Inter}[
    UprightFont = *-Regular,
    BoldFont = *-SemiBold,
    ItalicFont = *-Italic,
    BoldItalicFont = *-SemiBoldItalic
]
\setsansfont{Inter}[
    UprightFont = *-Regular,
    BoldFont = *-SemiBold,
    ItalicFont = *-Italic,
    BoldItalicFont = *-SemiBoldItalic
]

% ========== PAGE LAYOUT ==========
% Instructions: Adjust measurements to change margins and spacing
\usepackage{geometry}
\geometry{
    top=1.2in,
    bottom=1.2in,
    left=1.1in,
    right=1.1in,
    headheight=14pt,
    headsep=0.3in,
    footskip=0.4in
}

% ========== COLORS ==========
% Instructions: Modify these color definitions to match your brand/theme
\definecolor{primary}{HTML}{0F172A}        % Main text color - change as needed
\definecolor{secondary}{HTML}{334155}      % Secondary text color
\definecolor{accent}{HTML}{2563EB}         % Links and highlights
\definecolor{lightaccent}{HTML}{3B82F6}    % Lighter accent variant
\definecolor{success}{HTML}{059669}        % Success/positive elements
\definecolor{warning}{HTML}{D97706}        % Warning/attention elements
\definecolor{background}{HTML}{F8FAFC}     % Light background for boxes
\definecolor{border}{HTML}{E2E8F0}        % Borders and lines
\definecolor{textgray}{HTML}{64748B}      % Muted text
\definecolor{lightgray}{HTML}{F1F5F9}     % Table alternating rows
\definecolor{codebg}{HTML}{F8F8F8}        % Code block background
\definecolor{codecomment}{HTML}{6A737D}   % Code comments
\definecolor{codestring}{HTML}{032F62}    % Code strings
\definecolor{codekey}{HTML}{D73A49}       % Code keywords

% ========== TYPOGRAPHY ==========
% Instructions: These settings control paragraph spacing and text optimization
\usepackage{microtype}
\usepackage{parskip}
\setlength{\parindent}{0pt}
\setlength{\parskip}{8pt plus 2pt minus 1pt}

% ========== ADVANCED TEXT FEATURES ==========
% Instructions: For elegant drop caps at section beginnings
\usepackage{lettrine}
% Configuration for lettrine (drop caps)
\renewcommand{\LettrineFontHook}{\color{accent}\bfseries}
\setcounter{DefaultLines}{3}
\renewcommand{\DefaultLoversize}{0.1}
\renewcommand{\DefaultLraise}{0}

% Instructions: For proper quotation handling and formatting
\usepackage[autostyle=true,german=quotes,english=american]{csquotes}
\MakeOuterQuote{"}

% Instructions: For better paragraph flow control
\usepackage[all]{nowidow}
\usepackage{needspace}

% ========== SECTION STYLING ==========
% Instructions: Customize section headings appearance here
\usepackage{titlesec}
\titleformat{\section}
    {\color{primary}\fontsize{16}{20}\bfseries\sffamily}
    {\color{accent}\thesection}
    {1em}
    {}
    [\vspace{2pt}\color{border}\hrule height 0.8pt\vspace{6pt}]

\titleformat{\subsection}
    {\color{secondary}\fontsize{14}{18}\bfseries\sffamily}
    {\color{accent}\thesubsection}
    {0.8em}
    {}
    [\vspace{4pt}]

\titleformat{\subsubsection}
    {\color{secondary}\fontsize{12}{16}\bfseries\sffamily}
    {\color{accent}\thesubsubsection}
    {0.6em}
    {}

\titlespacing*{\section}{0pt}{20pt plus 4pt minus 2pt}{8pt plus 2pt minus 2pt}
\titlespacing*{\subsection}{0pt}{16pt plus 3pt minus 2pt}{6pt plus 2pt minus 1pt}
\titlespacing*{\subsubsection}{0pt}{12pt plus 2pt minus 1pt}{4pt plus 1pt minus 1pt}

% ========== HEADERS AND FOOTERS ==========
% Instructions: Replace "Your Document Title" and footer text with your content
\usepackage{fancyhdr}
\pagestyle{fancy}
\fancyhf{}
\renewcommand{\headrulewidth}{0.5pt}
\renewcommand{\footrulewidth}{0pt}
\fancyhead[L]{\color{textgray}\small\sffamily Your Document Title Here}
\fancyhead[R]{\color{textgray}\small\sffamily\thepage}
\fancyfoot[C]{\color{textgray}\footnotesize\sffamily Document Footer Text Here}

% ========== TABLE OF CONTENTS ==========
% Instructions: These settings style the table of contents
\usepackage{tocloft}
\renewcommand{\cftsecleader}{\cftdotfill{\cftdotsep}}
\renewcommand{\cftsubsecleader}{\cftdotfill{\cftdotsep}}
\setlength{\cftbeforesecskip}{6pt}
\setlength{\cftbeforesubsecskip}{3pt}
\renewcommand{\cftsecfont}{\color{primary}\bfseries\sffamily}
\renewcommand{\cftsubsecfont}{\color{secondary}\sffamily}
\renewcommand{\cftsecpagefont}{\color{accent}\bfseries\sffamily}
\renewcommand{\cftsubsecpagefont}{\color{accent}\sffamily}

% ========== ADVANCED REFERENCE MANAGEMENT ==========
% Instructions: For sophisticated bibliography management
\usepackage[
    backend=biber,
    style=authoryear,
    sorting=nyt,
    maxbibnames=99,
    giveninits=true
]{biblatex}
% If you have a bibliography file, uncomment and specify it here:
% \addbibresource{references.bib}

% Instructions: For page-aware references
\usepackage{varioref}
\labelformat{figure}{Figure~#1}
\labelformat{table}{Table~#1}
\labelformat{equation}{Equation~#1}
\labelformat{section}{Section~#1}

% ========== PAGE BREAK CONTROL ==========
% Instructions: For better page break control
\usepackage{afterpage}

% ========== BOXES AND CALLOUTS ==========
% Instructions: Required TikZ libraries for shadows in boxes
\usepackage{tikz}
\usetikzlibrary{shadows}

% Instructions: Three types of callout boxes are defined below
\usepackage[most]{tcolorbox}

% Info box style - use for general information
\newtcolorbox{infobox}{
    colback=background,
    colframe=accent,
    coltext=primary,
    boxrule=1pt,
    arc=3pt,
    left=12pt,
    right=12pt,
    top=8pt,
    bottom=8pt,
    enhanced,
    drop shadow,
    fonttitle=\bfseries\sffamily\color{accent},
    title style={left color=accent!10, right color=accent!5},
    attach boxed title to top left={xshift=8pt, yshift=-2pt}
}

% Alert box style - use for cautions and important notes
\newtcolorbox{alertbox}{
    colback=warning!5,
    colframe=warning,
    coltext=primary,
    boxrule=1pt,
    arc=3pt,
    left=12pt,
    right=12pt,
    top=8pt,
    bottom=8pt,
    enhanced,
    drop shadow
}

% Success box style - use for positive highlights
\newtcolorbox{successbox}{
    colback=success!5,
    colframe=success,
    coltext=primary,
    boxrule=1pt,
    arc=3pt,
    left=12pt,
    right=12pt,
    top=8pt,
    bottom=8pt,
    enhanced,
    drop shadow
}

% ========== LOGO-ENHANCED BOXES ==========
\usepackage{bclogo}

% ========== HYPERLINKS ==========
% Instructions: Update PDF metadata and link colors as needed
\usepackage{hyperref}
\hypersetup{
    colorlinks=true,
    linkcolor=accent,
    urlcolor=lightaccent,
    citecolor=accent,
    filecolor=accent,
    bookmarks=true,
    bookmarksopen=true,
    bookmarksnumbered=true,
    pdfstartview=FitH,
    pdfpagelayout=SinglePage,
    pdftitle={Your PDF Title Here},
    pdfauthor={Your Name Here},
    pdfsubject={Document Subject Here},
    pdfkeywords={keyword1, keyword2, keyword3}
}

% Instructions: For intelligent cross-referencing
\usepackage{cleveref}
\crefname{figure}{Figure}{Figures}
\crefname{table}{Table}{Tables}
\crefname{equation}{Equation}{Equations}
\crefname{section}{Section}{Sections}

% ========== LISTS ==========
% Instructions: These settings control list appearance and spacing
\usepackage{enumitem}
\setlist[itemize,1]{
    leftmargin=18pt,
    itemsep=3pt plus 1pt minus 1pt,
    parsep=0pt,
    topsep=6pt plus 2pt minus 2pt,
    label=\color{accent}$\bullet$
}
\setlist[itemize,2]{
    leftmargin=16pt,
    itemsep=2pt,
    parsep=0pt,
    topsep=3pt,
    label=\color{secondary}$\circ$
}

% ========== TABLES ==========
% Instructions: Packages for professional table formatting
\usepackage{booktabs}
\usepackage{tabularx}
\usepackage{caption}

% Table styling
\captionsetup{font=small, labelfont=bf, labelsep=colon}

% Additional table colors
\definecolor{cardbg}{HTML}{F7F9FC}
\definecolor{headerbg}{HTML}{4A90E2}
\definecolor{headertext}{HTML}{FFFFFF}
\definecolor{rowalt}{HTML}{F1F4F9}
\definecolor{frameborder}{HTML}{D3DCE6}

% ========== CODE LISTINGS ==========
% Instructions: For code syntax highlighting
\usepackage{listings}
\lstset{
    basicstyle=\ttfamily\small,
    backgroundcolor=\color{codebg},
    commentstyle=\color{codecomment},
    keywordstyle=\color{codekey}\bfseries,
    stringstyle=\color{codestring},
    numbers=left,
    numberstyle=\tiny\color{textgray},
    numbersep=10pt,
    tabsize=4,
    breaklines=true,
    breakatwhitespace=false,
    frame=single,
    rulecolor=\color{border},
    framesep=8pt,
    showstringspaces=false
}

% Uncomment for advanced syntax highlighting with minted (requires Python)
\usepackage{minted}
\setminted{
    style=default,
    bgcolor=codebg,
    linenos=true,
    breaklines=true,
    tabsize=4,
    fontsize=\small,
    frame=single,
    framesep=8pt
}

% ========== ALGORITHMS ==========
% Instructions: For pseudocode or process descriptions
\usepackage[ruled,vlined]{algorithm2e}
\SetAlgoLined
\SetAlgoSkip{medskip}
\SetAlFnt{\footnotesize\sffamily}
\renewcommand{\algorithmcfname}{Algorithm}

% ========== ICON BOXES ==========
% Load fontawesome and awesomebox for icon support
\usepackage{fontawesome5}
\usepackage{awesomebox}

% ========== TITLE PAGE ==========
% Instructions: Controls the appearance of the title page
\usepackage{titling}
\pretitle{
    \begin{center}
    \vspace{0.5in}
    \color{primary}\Huge\bfseries\sffamily
}
\posttitle{
    \end{center}
    \vspace{0.3in}
    \begin{center}
    \color{border}\rule{0.6\textwidth}{2pt}
    \end{center}
    \vspace{0.2in}
}
\preauthor{\begin{center}\color{secondary}\large\sffamily}
\postauthor{\end{center}}
\predate{\begin{center}\color{textgray}\large\sffamily}
\postdate{\end{center}\vspace{0.5in}}

% ========== DOCUMENT METADATA ==========
% Instructions: Replace with your actual document information
\title{Your Main Document Title Here\\[0.3em]Optional Subtitle Goes Here}
\author{Author Name or Organization}
\date{\today} % This will show current date, or replace with specific date 