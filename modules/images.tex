% ========== IMAGES MODULE ==========
% Load this module only when advanced image features are needed

% Image packages
\usepackage{graphicx}
\usepackage[export,graphics,noclipbox]{adjustbox}
\usepackage{varwidth}
\usepackage{tikz}

% Create fallback box for missing images
\newsavebox{\imagefallbackbox}

% Command for images with fallback text
\newcommand{\imageWithFallback}[4][]{%
  \begingroup
  \sbox{\imagefallbackbox}{%
    \begin{varwidth}{\textwidth}
      \centering
      \textcolor{textgray}{\textbf{#4}}
    \end{varwidth}
  }%
  \IfFileExists{#2}{%
    \includegraphics[#1,width=#3]{#2}%
  }{%
    \fcolorbox{border}{background}{%
      \begin{minipage}[c][0.75\ht\imagefallbackbox][c]{#3}
        \centering
        \textcolor{textgray}{\faImage}\\[0.3em]
        \usebox{\imagefallbackbox}
      \end{minipage}%
    }%
  }%
  \endgroup
}

% Rectangular image with border
\newcommand{\rectImage}[5][]{%
  \begin{figure}[htbp]
    \centering
    \fbox{\imageWithFallback[#1]{#2}{#3}{#5}}
    \ifx&#4&%
    \else
      \caption{#4}
    \fi
  \end{figure}
}

% Rounded corner image
\newcommand{\roundedImage}[6][]{%
  \begin{figure}[htbp]
    \centering
    \begin{tikzpicture}
      \node[inner sep=0pt] (image) {%
        \imageWithFallback[#1]{#2}{#3}{#6}%
      };
      \begin{scope}
        \clip[rounded corners=#4] (image.south west) rectangle (image.north east);
        \node[inner sep=0pt] {\imageWithFallback[#1]{#2}{#3}{#6}};
      \end{scope}
    \end{tikzpicture}
    \ifx&#5&%
    \else
      \caption{#5}
    \fi
  \end{figure}
}

% Circular image
\newcommand{\circularImage}[5][]{%
  \begin{figure}[htbp]
    \centering
    \begin{tikzpicture}
      \node[inner sep=0pt] (image) {%
        \imageWithFallback[#1]{#2}{#3}{#5}%
      };
      \begin{scope}
        \clip (image.center) circle (#3/2);
        \node[inner sep=0pt] {\imageWithFallback[#1]{#2}{#3}{#5}};
      \end{scope}
    \end{tikzpicture}
    \ifx&#4&%
    \else
      \caption{#4}
    \fi
  \end{figure}
}

% Image with drop shadow
\newcommand{\shadowImage}[5][]{%
  \begin{figure}[htbp]
    \centering
    \begin{tikzpicture}
      \node[inner sep=0pt, drop shadow={shadow xshift=2pt, shadow yshift=-2pt, opacity=0.5}] {%
        \imageWithFallback[#1]{#2}{#3}{#5}%
      };
    \end{tikzpicture}
    \ifx&#4&%
    \else
      \caption{#4}
    \fi
  \end{figure}
}

% Full-width responsive image
\newcommand{\fullwidthImage}[5][]{%
  \begin{figure}[htbp]
    \centering
    \makebox[\textwidth][c]{%
      \imageWithFallback[#1,height=#3,width=\textwidth,keepaspectratio]{#2}{\textwidth}{#5}%
    }
    \ifx&#4&%
    \else
      \caption{#4}
    \fi
  \end{figure}
}
